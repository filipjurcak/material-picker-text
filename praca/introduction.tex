\chapter*{Introduction} % chapter* je necislovana kapitola
\addcontentsline{toc}{chapter}{Introduction} % rucne pridanie do obsahu
\markboth{Introduction}{Introduction} % vyriesenie hlaviciek

Setting material appearance is one of the most crucial steps in modeling 3D scenes and probably the most important one for creating a realistic model look. This task is often long, tedious, and requires non-trivial skill, as a lot of parameters need and can be set up for a model to look realistic after rendering. The number of parameters varies between used material models, but realistic models often need tens of correctly set up parameters.
\newline
As a result of this thesis, we want to ease the whole process by providing artists with a material picker tool. This tool is a deep neural network that would estimate a lot of intrinsic properties of an image, which would help us recover material from a user-specified object in an image.
\newline
We do so by inventing a pipeline for solving two fundamental problems in computer graphics and computer vision - inverse rendering and material segmentation, both from a single image. In addition to segmenting the input image, our method performs per-pixel estimation of the number of intrinsic scene characteristics, such as diffuse and specular albedo, surface normals, glossiness, and view vector. We use all of these inferred properties to simulate the rendering process, yielding close approximation of an input image.
\newline
To train all models in our pipeline, we present a new way to create a modern dataset by using advanced features of a physically-based V-Ray renderer to bridge the gap between synthetic data and real images. This is crucial, as we most often want to generalize well on real images, which is hard to achieve with synthetic images only.